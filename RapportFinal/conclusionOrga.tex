\documentclass[a4paper,11pt]{article}
\usepackage[utf8]{inputenc}
\usepackage[T1]{fontenc}
\usepackage[french]{babel}
\usepackage[right=2.5cm, left=2.5cm, bottom=4cm, top=3cm]{geometry}
\usepackage{textcomp}
\usepackage{graphicx}
\usepackage{mathtools,amssymb,amsthm}
\usepackage{lmodern}
\usepackage{multirow}
\usepackage{array}
\usepackage{longtable}

\title{\vspace{13em}{\huge Rapport Final}}
\author{Edouard Fouassier - Maxime Gonthier - Benjamin Guillot\\
		Laureline Martin - Rémi Navarro - Lydia Rodrigez de la Nava
		\vspace{2em}\\
		Algorithme Génétique
		\vspace{2em}}

\begin{document}
	
	\pagenumbering{gobble}\clearpage
	\maketitle\vspace{13em}
\newpage
\tableofcontents
\newpage\clearpage\pagenumbering{arabic}
	\section{Conclusion sur l'organisation}
		En termes d’organisation, nous avons mis en place deux rendez-vous par semaines pendant lesquels le groupe entier était présent.\\
		Le premier était prévu tous les mercredis, juste après notre entrevue avec Mme Kloul, pendant lequel nous nous répartissions les tâches et travaillons sur les remarques faites précédemment par la professeure.\\
		Nous nous retrouvions ensuite le mardi suivant pour mettre en commun toutes nos recherches et nous mettre d’accord sur nos choix pour notre projet selon le planning, et sur notre présentation du lendemain. Le reste du temps nous restions en contact via un groupe de messagerie, nous partagions nos recherches sur Google Document et sur GitHub. Lors de l’implémentation, nous avons séparé le groupe en binôme, chacun était chargé d’un module, tout en restant vigilant des autres modules.\\
		Malgré cette organisation, il y a eu des semaines pour lesquelles nous avions sous-estimé la complexité de certaines tâches, et pendant lesquelles nos réunions ne servaient non plus à rassembler les informations mais à les rechercher. Ces changements de dernières minutes nous ont poussé à nous diviser pour effectuer les recherches. Ce qui a pris de notre temps pour nous mettre d’accord et prendre des décisions.\\
		Cela a conduit à des réunions où tout le monde n’était plus concentré sur le même problème, mais plutôt sur plusieurs, ce qui a conduit à un certain manque de communication puisque tout le monde n’était pas attentif lorsqu’une décision était prise.\\
 		Il est possible que notre manque de recherches et les incohérences prennent leurs sources dans ce problème-là.\\
		Nous sommes conscients des faiblesses dans notre projet mais sortons grandis de l’expérience. Nous savons que les erreurs qui ont été faites ne seront plus reproduites dans nos futurs projets à chacun, nous avons appris beaucoup en terme d’organisation et l’importance d’une bonne cohésion de groupe.\\ 
		Chacun de nous a découvert quels sont ses points forts et ses points faibles, et nous ferons de notre mieux pour travailler sur nos points faibles et d’exploiter au mieux nos points forts. Ce projet nous a aussi appris l’importance de travailler de manière suffisamment intelligible pour que tous les membres du groupe comprennent une notion ou un morceau de code.\\
\end{document}







   
